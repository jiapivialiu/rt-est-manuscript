\section{Introduction}
\label{sec:intro}

%% introduce Rt 
Effective reproduction numbers (or instantaneous reproduction numbers) are a key to understand infectious disease dynamics including the potential sizes of a pandemic, the scale of epidemic prevention measures such as the proportion of a population that should be vaccinated, and the effectiveness of control effects. 
They are expected numbers of secondary infections caused by an infected individual in a population. It is a time series that can reflect the effect of time-varying factors, such as intervention policy and population immunity. 
Let $\calR_t$ be the effective reproduction number at time $t$. A practical interpretation is that $\calR_t < 1$ represents a circumstance when the infection dies out gradually and achieves a \textit{disease-free equilibrium}, whereas when $\calR_t > 1$, the infection is always present in a state of \textit{endemic equilibrium}. 
%
%% significance/necessity of modeling Rt: 
Effective reproduction number reveals an unobservable biological reality. 
%% properties of interest 
There exist a great number of models to uncover this reality relying on various domain-specific assumptions and using different types of observed data such as incidence data and death rates. 
%It, on one hand, implies that the effective reproduction number estimation of a pathogen can vary. 
Different estimation processes and results suggest different aspects and levels of policy making in epidemic control. For example, if effective reproduction numbers are assumed to be affected by human-human interaction over time in an epidemic model, controlling the parameters influencing human-human contact rate may eventually lead to a reduction of the effective reproduction numbers in the specific model. 
Since some assumptions cannot be verified in practice, it is critical for an estimator to be \textit{robust} to model misspecification. 
The estimation relies heavily on the quality of the available data. 
The infectious data may contain poor information due to the limitations of data collection, which results in difficulties in reproduction number estimation. 
%Thus, it is important to make \textit{accurate} estimation that is \textit{robust} to poor condition of data such as the low incidence periods. 
%% usage: retrospective or real-time estimation, and prediction

%% literature review: Bayesian approaches: EpiEstim, EpiFilter, EpiNow2, EpiNowCast
Many existing approaches for effective reproduction number estimation are Bayesian approaches that estimate posterior distribution of $\calR_t$. 
\cite{cori2013new} proposed a Bayesian approach, EpiEstim, to solve the posterior distribution of $\calR_t$ given incidence data prior to time $t$. An advantage of EpiEstim is that it depends on limited assumptions (the Poisson distributed incidences and Gamma distribution effective reproduction number) and only requires incidence data that is easily obtainable. For these reasons, EpiEstim does not require much domain expertise in implementation. It is one of the earliest approaches that are both succinct and accurate in $\calR_t$ estimation. They proposed EpiEstim(2.2) in \cite{thompson2019improved}, which distinguished imported cases from local transmission and directly estimated the serial interval. They further extended EpiEstim by using reconstructed daily incidence data to overcome the issue when incidences are not always daily records in \cite{nash2023estimating}. 
% bayEstim is a version of implementation of {cori2013new} and {thompson2019improved}. 
% 
\cite{parag2021improved} proposed an alternative Bayesian approach, EpiFilter, that is a recursive Bayesian smoother based on Kalman Filter. EpiFilter also solves the posteriori of $\calR_t$ given a Gamma prior and Poisson distributed infection counts. Compared to EpiEstim, EpiFilter estimates $\calR_t$ retrospectively using all available incidences and generates more robust estimation in low incidence periods and accurate one-step-ahead prediction. 
%koyama2021estimating: devised a state-space method for fitting a discrete-time variant of the Hawkes process to a given dataset of daily
\cite{abbott2020estimating} proposed a Bayesian latent variable framework, EpiNow2, which uses both incidence and death counts and provides precise $\calR_t$ estimation and forecasts. They further proposed a generative Bayesian model to handle missing data by imputation followed by truncation adjustment in \cite{lison2023generative}. 
\cite{gressani2022epilps} proposed a Bayesian P-splines approach, EpiLPS, and argued that this approach was computationally efficient. They assumed the incidence cases follow a negative binomial distribution. 
\cite{trevisin2023spatially} proposed a Bayesian model based on particle filtering to estimate spatially explicit effective reproduction numbers. 
%
Bayesian approaches estimate the posterior distribution of the effective reproduction numbers with the advantage that the credible interval estimation can be easily computed. A limitation of many Bayesian approaches, however, is that they usually require heavy computational workload, especially when data sequences are long or hierarchical structures are complex. 
%A similarity of these models is that they only require the knowledge of the limited information, such as infected counts and the distribution of serial interval, to make accurate estimation. 

\cite{abry2020spatial} proposed to regularize the smoothness of $\calR_t$ regarding its temporal and spatial evolution. They considered a penalized regression with a second-order temporal regularization and a spatial regularization on $\calR_t$ and with Poisson loss. They further extended it by introducing another penalty on outliers for robustness in \cite{pascal2022nonsmooth}. 
%
There are other spline-based approaches such as \cite{azmon2014estimation,gressani2021approximate,pircalabelu2023spline}. %and compartmental models. 
\cite{jin2023epimix} proposed EpiMix, which included exogenous factors other than incidence cases and introduced random effects in regression model. It is also robust against poor data condition, i.e., low incidence. 
%
There are many other practical considerations in effective reproduction number estimation. \cite{hettinger2023estimating} proposed a generalized autoregresive moving average (GARMA) model. It solved the bias introduced by practical concerns such as particular forms of measurement errors in incidence data. They also incorporates multiple serial interval distributions. %Basically extended EpiEstim. 
%
\cite{pircal2023spline} is a frequentist method and based on splines. They assumed the incidence follow an exponential-family distribution. %but in a more continuous-time setting. [Important!]
\cite{ho2023accounting} considered overdispersion by making a negative binomial assumption to estimate Covid-19 $\calR_t$. 

%%%%%%%%%%%%%%%%%%%%%%%%%%%%%%%%%%%%%%%%%% our approach %%%%%%%%%%%%%%%%%%%%%%%%%%%%%%%%%%%%%%%%%%
We propose a retrospective effective reproduction number estimation approach to filter out discrete splines. More specifically, it is an optimization problem with Poisson loss and $\ell_1$ penalty on the temporal evolution of $\calR_t$, which is known as the trend filtering penalty \citep{kim2009ell_1,tibshirani2014adaptive}. The estimated curves appear to be piecewise polynomial. The estimators have both local adaptivity (i.e., heterogeneous smoothness throughout the range of time) and allow the computational efficiency. Similar as the aforementioned approaches, our approach only requires limited assumptions on the distribution of incidences and serial intervals, and depends on the widely obtainable incidence data. Thus, it only requires limited domain expertise to apply our approach. We follow the common assumption of Poisson distributed incidence data, but the empirical study shows that the estimators are robust in negative binomial settings. 
% the approach we focus
Our effective reproduction number estimation is accurate and robust in model misspecification and computationally efficient compared to the aforementioned competitors. 
In computation, our approach takes the advantage of convex optimization and is solved by Newton's method, which is known to converge rapidly. Moreover, the sparse structure of the divided difference matrix used in the temporal penalty allows further efficiency in computation. 
%Temporal evolution of reproduction numbers considers the progression of reproduction numbers that occurs over time through different temporal intervals. 

%%%%%%%%%%%%%%%%%%%%%%%%%%%%%%%%%%%%%%%%%% overview %%%%%%%%%%%%%%%%%%%%%%%%%%%%%%%%%%%%%%%%%%
