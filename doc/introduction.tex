\section{Introduction}
\label{sec:intro}

%% introduce Rt 
The effective reproduction number (also called the instantaneous reproduction
number) is a key quantity for understanding infectious disease dynamics
including the potential size of a pandemic, the required stringency of
prevention measures, and the efficacy of other controls. The effective
reproduction number is defined to be the average number of secondary infections
caused by a new primary infection that occurs at a specific time. Tracking the
time series of this quantity is therefore useful for understanding whether or
not future infections are likely to increase or decrease from the current state.
Let $\calR(t)$ denote the effective reproduction number at time $t$.
Practically, as long as $\calR(t) < 1$, infections will decline gradually,
eventually resulting in a disease-free equilibrium, whereas when $\calR(t) > 1$,
infections will continue to increase, resulting in endemic equilibrium. 
%% significance/necessity of modeling Rt: 
While $\calR(t)$ is fundamentally a continuous time quantity, it can be related
to data only at discrete points in time $t = 1,\ldots,n$.
This sequence of effective reproduction numbers over time is not observable, but,
nonetheless, is easily interpretable and retrospectively describes the course of
an epidemic. Therefore, a number of procedures exist to estimate $\calR_t$ from
different types of observed incidence data such as cases, deaths, or
hospitalizations, while relying on various domain-specific assumptions.
%% properties: accurate and robust to violation of assumptions 
Importantly, accurate estimation of effective reproduction numbers relies
heavily on the quality of the available data, and, due to the limitations of
data collection, such as underreporting and lack of standardization,
estimation methodologies rely on various assumptions to
compensate. Because model assumptions may not be easily verifiable from data
alone, it is also critical for any estimation procedure to be robust to model
misspecification. 

%% literature review: Bayesian approaches: EpiEstim, EpiFilter, EpiNow2, EpiNowCast
Many existing approaches for effective reproduction number estimation are
Bayesian: they estimate the posterior distribution of $\calR_t$ conditional on
the observations. One of the first such approaches is the software \EpiEstim\
\citep{cori2020package}, described by \citet{cori2013new}. This method is
prospective, in that it uses only observations available up to time $t$ in order to
estimate $\calR_t$ for each $i = 1,\ldots, t$. An advantage of \EpiEstim\ is its
straightforward statistical model: new incidence data follows the Poisson
distribution conditional on past incidence combined with the conjugate gamma prior
distribution for $\calR_t$ with fixed hyperparameters. Additionally, the serial
interval distribution is fixed and known. For this reason, \EpiEstim\ requires
little domain expertise for use, and it is computationally fast.
\citet{thompson2019improved} modified this method to distinguish imported cases from local transmission and
simultaneously estimate the serial interval distribution.
\citet{nash2023estimating} further extended \EpiEstim\ by using
``reconstructed'' daily incidence data to handle irregularly spaced observations.
% 
\cite{abbott2020estimating} proposed a Bayesian latent variable framework,
\texttt{EpiNow2} \citep{EpiNow2}, which leverages incident cases, deaths or
other available streams simultaneously along with allowing additional delay
distributions (incubation period and onset to reporting delays) in modelling.  
\citet{lison2023generative} proposed an extension that handles missing data by
imputation followed by a truncation adjustment. These modifications are intended
to increase accuracy at the most recent (but most uncertain) timepoints, to aid policymakers.
%
\citet{parag2021improved} also proposed a Bayesian approach, \texttt{EpiFilter}
based on the (discretized) Kalman filter and smoother. \texttt{EpiFilter} also
estimates the posterior of $\calR_t$ given a Gamma prior and Poisson distributed
incident cases. Compared to \EpiEstim, however, \texttt{EpiFilter} estimates
$\calR_t$ retrospectively using all available incidence data both before and
after time $t$, with the goal of being more robust in low-incidence cases.  
\citet{gressani2022epilps} proposed a Bayesian P-splines approach, \EpiLPS, that
assumes negative Binomial distributed incidence. \citet{trevisin2023spatially}
also proposed a Bayesian model estimated with particle filtering to incorporate
spatial structures.
%
Bayesian approaches estimate the posterior distribution of the effective
reproduction numbers and possess the advantage that credible intervals may be
easily computed. A limitation of many Bayesian approaches, however, is that they
usually require more intensive computational routines, especially when observed
data sequences are long or hierarchical structures are complex.  Below, we
compare our method to two of the more computationally efficient Bayesian models,
\EpiEstim\ and \EpiLPS. 

There are also frequentist approaches for $\calR_t$ estimation.
\citet{abry2020spatial} proposed regularizing the smoothness of $\calR_t$
through penalized
regression with second-order temporal regularization, additional spatial
penalties, and with Poisson loss. \citet{pascal2022nonsmooth}
extended this procedure by introducing another penalty on outliers.
%
\cite{pircalabelu2023spline} proposed a spline-based model relying on the 
assumption of exponential-family distributed incidence. 
\cite{ho2023accounting} estimates $\calR_t$ while monitoring the time-varying
level of overdispersion. 
%
There are other spline-based approaches such as
\cite{azmon2014estimation,gressani2021approximate},
autoregressive models with random effects \citep{jin2023epimix} that are robust
to low incidence cases, and generalized autoregressive moving average (GARMA)
models \citep{hettinger2023estimating} that are robust to measurement errors in
incidence data. 


%%%%%%%%%%%%%%%%%%%%%%%%%%%%%%%%%%%%%%%%%% our approach %%%%%%%%%%%%%%%%%%%%%%%%%%%%%%%%%%%%%%%%%%
We propose a retrospective effective reproduction number estimator
called \RtEstim\ that requires only incidence data. Our model makes the
conditional Poisson assumption, similar to much of the prior work described
above, but is empirically more robust to misspecification. This estimator is defined by a
convex optimization problem with Poisson loss and $\ell_1$ penalty on the
temporal evolution of $\log(\calR_t)$ to impose smoothness over time. 
As a result, \RtEstim\ generates discrete splines, and the estimated curves (in
$\log$-space) appear to be piecewise polynomials of an order selected by the
user. Importantly, The estimates are locally adaptive, meaning that different
time ranges may posses heterogeneous smoothness. Because we penalize the
logarithm of $\calR_t$, we naturally accommodate the positivity requirement, can
handle large or small incidence measurements, and are automatically (reasonably)
robust to outliers without additional constraints. A small illustration using
three years of Covid-19 case data in British Columbia is shown in \autoref{fig:intro-fig}.

\begin{figure}[tb]
    \centering
    \includegraphics[width=.99\textwidth]{fig/intro-fig-new.png}
    \caption{A demonstration of effective reproduction number estimation 
    by \RtEstim\ and the corresponding fitted incidence cases for the Covid-19 epidemic 
    in British Columbia, Canada during a period from March 28, 2020 to June 28, 2023. 
    The \textcolor{customblue}{\textbf{blue}} curve in the top panel is the estimated piecewise
    quadratic $\calR_t$ and the gray ribbon is the corresponding 95\% confidence band. 
    The black curve in the bottom panel is the observed Covid-19 daily confirmed 
    cases, and the \textcolor{customorange}{\textbf{orange}} curve is the fitted incidence 
    corresponding to the estimated $\calR_t$.}
    \label{fig:intro-fig}
\end{figure}

% Due to the limitations including but not limited to
% insufficient surveillance resources and incomplete reporting, the incidence
% cases are only observed to a certain proportion of the unknown, true values.
% Here, we assume that this proportion is smoothly time-varying, which will not
% cause an immediate changing point in the pattern of transmissibility of
% epidemics. With this assumption, we argue that the observable incidence contain
% the true curvature and changing points of the underlying effective reproduction
% numbers of epidemics. We consider a fixed serial interval distribution that
% needs to be pre-specified. It can either be chosen parameters of Gamma
% distribution or a series of probabilities provided by related methods with sum
% $1$. Serial intervals have been studied for specific infectious diseases, such
% as H1N1 influenza and Covid-19, by numerous existing studies
% \citep{white2009estimation,boelle2011transmission,rai2021estimates,alene2021serial,griffin2020rapid}. 
While our approach is straightforward and requires little domain knowledge for
implementation, we also implement a number of refinements. 
% \RtEstim\ produces accurate estimations that are empirically robust in model misspecification, i.e., the violation of distributional assumption of incidence. 
% 
 %We follow the common assumption of Poisson distributed incidence data, but the empirical study shows that the estimators are robust in negative binomial settings. 
%Thus, RtEstim is accurate, robust in model misspecification, and computationally efficient. 
% the approach we focus 
We use a proximal Newton method to solve the convex optimization problem along
with warm starts to produce estimates efficiently,
typically in a matter of seconds, even for long sequences of data.
%  Our approach takes the advantage of convex optimization and is
% solved by
% Newton's method, which is known to converge rapidly. %We provide a more
% computationally efficient algorithm for the lowest-degree problem, where
% iterative algorithms can be avoided. 
% Moreover, the sparse structure of the divided difference matrix used in the trend filtering penalty allows further efficiency in computation. 
%Temporal evolution of reproduction numbers considers the progression of reproduction numbers that occurs over time through different temporal intervals. 
In a number of simulated experiments, we show empirically that our approach is
more accurate than existing methods at estimating the true effective reproduction numbers. 


The manuscript proceeds as follows. We first introduce the methodology of
\RtEstim\ including the usage of renewal equation and the development of Poisson
trend filtering estimator. We explain how this method could be interpreted from
the Bayesian perspective, connecting it to previous work in this context. We
provide illustrative experiments comparing our estimator to \EpiEstim\ and
\EpiLPS. We then apply our \RtEstim\ on the Covid-19 pandemic incidence in
British Columbia and the 1918 influenza pandemic in the United States. Finally,
we conclude with a discussion of the advantages and limitations of our approach
and describe practical considerations for effective reproduction number
estimation.
