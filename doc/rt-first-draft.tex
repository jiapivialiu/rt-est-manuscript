% Template for PLoS
% Version 3.6 Aug 2022
%
% % % % % % % % % % % % % % % % % % % % % %
%
% -- IMPORTANT NOTE
%
% This template contains comments intended 
% to minimize problems and delays during our production 
% process. Please follow the template instructions
% whenever possible.
%
% % % % % % % % % % % % % % % % % % % % % % % 
%
% Once your paper is accepted for publication, 
% PLEASE REMOVE ALL TRACKED CHANGES in this file 
% and leave only the final text of your manuscript. 
% PLOS recommends the use of latexdiff to track changes during review, as this will help to maintain a clean tex file.
% Visit https://www.ctan.org/pkg/latexdiff?lang=en for info or contact us at latex@plos.org.
%
%
% There are no restrictions on package use within the LaTeX files except that no packages listed in the template may be deleted.
%
% Please do not include colors or graphics in the text.
%
% The manuscript LaTeX source should be contained within a single file (do not use \input, \externaldocument, or similar commands).
%
% % % % % % % % % % % % % % % % % % % % % % %
%
% -- FIGURES AND TABLES
%
% Please include tables/figure captions directly after the paragraph where they are first cited in the text.
%
% DO NOT INCLUDE GRAPHICS IN YOUR MANUSCRIPT
% - Figures should be uploaded separately from your manuscript file. 
% - Figures generated using LaTeX should be extracted and removed from the PDF before submission. 
% - Figures containing multiple panels/subfigures must be combined into one image file before submission.
% For figure citations, please use "Fig" instead of "Figure".
% See http://journals.plos.org/plosone/s/figures for PLOS figure guidelines.
%
% Tables should be cell-based and may not contain:
% - spacing/line breaks within cells to alter layout or alignment
% - do not nest tabular environments (no tabular environments within tabular environments)
% - no graphics or colored text (cell background color/shading OK)
% See http://journals.plos.org/plosone/s/tables for table guidelines.
%
% For tables that exceed the width of the text column, use the adjustwidth environment as illustrated in the example table in text below.
%
% % % % % % % % % % % % % % % % % % % % % % % %
%
% -- EQUATIONS, MATH SYMBOLS, SUBSCRIPTS, AND SUPERSCRIPTS
%
% IMPORTANT
% Below are a few tips to help format your equations and other special characters according to our specifications. For more tips to help reduce the possibility of formatting errors during conversion, please see our LaTeX guidelines at http://journals.plos.org/plosone/s/latex
%
% For inline equations, please be sure to include all portions of an equation in the math environment.  For example, x$^2$ is incorrect; this should be formatted as $x^2$ (or $\mathrm{x}^2$ if the romanized font is desired).
%
% Do not include text that is not math in the math environment. For example, CO2 should be written as CO\textsubscript{2} instead of CO$_2$.
%
% Please add line breaks to long display equations when possible in order to fit size of the column. 
%
% For inline equations, please do not include punctuation (commas, etc) within the math environment unless this is part of the equation.
%
% When adding superscript or subscripts outside of brackets/braces, please group using {}.  For example, change "[U(D,E,\gamma)]^2" to "{[U(D,E,\gamma)]}^2". 
%
% Do not use \cal for caligraphic font.  Instead, use \mathcal{}
%
% % % % % % % % % % % % % % % % % % % % % % % % 
%
% Please contact latex@plos.org with any questions.
%
% % % % % % % % % % % % % % % % % % % % % % % %

\documentclass[10pt,letterpaper]{article}
\usepackage[top=0.85in,left=2.75in,footskip=0.75in]{geometry}

% amsmath and amssymb packages, useful for mathematical formulas and symbols
\usepackage{amsmath,amssymb}

% Use adjustwidth environment to exceed column width (see example table in text)
\usepackage{changepage}

% textcomp package and marvosym package for additional characters
\usepackage{textcomp,marvosym}

% cite package, to clean up citations in the main text. Do not remove.
\usepackage{cite}

% Use nameref to cite supporting information files (see Supporting Information section for more info)
\usepackage{nameref,hyperref}

% line numbers
\usepackage[right]{lineno}

% ligatures disabled
\usepackage[nopatch=eqnum]{microtype}
\DisableLigatures[f]{encoding = *, family = * }

% color can be used to apply background shading to table cells only
\usepackage[table]{xcolor}

% array package and thick rules for tables
\usepackage{array}

% create "+" rule type for thick vertical lines
\newcolumntype{+}{!{\vrule width 2pt}}

% create \thickcline for thick horizontal lines of variable length
\newlength\savedwidth
\newcommand\thickcline[1]{%
  \noalign{\global\savedwidth\arrayrulewidth\global\arrayrulewidth 2pt}%
  \cline{#1}%
  \noalign{\vskip\arrayrulewidth}%
  \noalign{\global\arrayrulewidth\savedwidth}%
}

% \thickhline command for thick horizontal lines that span the table
\newcommand\thickhline{\noalign{\global\savedwidth\arrayrulewidth\global\arrayrulewidth 2pt}%
\hline
\noalign{\global\arrayrulewidth\savedwidth}}


% Remove comment for double spacing
%\usepackage{setspace} 
%\doublespacing

% Text layout
\raggedright
\setlength{\parindent}{0.5cm}
\textwidth 5.25in 
\textheight 8.75in

% Bold the 'Figure #' in the caption and separate it from the title/caption with a period
% Captions will be left justified
\usepackage[aboveskip=1pt,labelfont=bf,labelsep=period,justification=raggedright,singlelinecheck=off]{caption}
\renewcommand{\figurename}{Fig}

% Use the PLoS provided BiBTeX style
\bibliographystyle{plos2015}

% Remove brackets from numbering in List of References
\makeatletter
\renewcommand{\@biblabel}[1]{\quad#1.}
\makeatother



% Header and Footer with logo
\usepackage{lastpage,fancyhdr,graphicx}
\usepackage{epstopdf}
%\pagestyle{myheadings}
\pagestyle{fancy}
\fancyhf{}
%\setlength{\headheight}{27.023pt}
%\lhead{\includegraphics[width=2.0in]{PLOS-submission.eps}}
\rfoot{\thepage/\pageref{LastPage}}
\renewcommand{\headrulewidth}{0pt}
\renewcommand{\footrule}{\hrule height 2pt \vspace{2mm}}
\fancyheadoffset[L]{2.25in}
\fancyfootoffset[L]{2.25in}
\lfoot{\today}

%% Include all macros below

\newcommand{\lorem}{{\bf LOREM}}
\newcommand{\ipsum}{{\bf IPSUM}}

%% END MACROS SECTION

%% use-defined citations & references colors
\definecolor{ceruleanblue}{rgb}{0.16, 0.32, 0.75}
\hypersetup{
    colorlinks=true,
    citecolor=ceruleanblue,
    linkcolor=ceruleanblue,
    urlcolor=ceruleanblue
}
%% end user-defined colors

%% user-defined math definitions
\def\calR{\mathcal{R}}
\def\bbN{\mathbb{N}}
\def\bbR{\mathbb{R}}
\def\bbP{\mathbb{P}}
\def\bbZ{\mathbb{Z}}
\DeclareMathOperator*{\diag}{diag}
\DeclareMathOperator*{\argmin}{argmin}
\newcommand{\Argmin}[1]{\underset{#1}{\argmin\ }}
\DeclareMathOperator*{\argmax}{argmax}
\newcommand{\Argmax}[1]{\underset{#1}{\argmax\ }}
\def\sumN{\sum_{i=1}^n}
\newcommand{\fr}[1]{\frac{1}{#1}}
\newcommand{\lr}[1]{\left(#1\right)}
\newcommand{\norm}[1]{\left\lVert #1 \right\rVert}
\def\Dxkk{D^{(x,k+1)}}
%% END math definitions

%% user-defined reference structures
\newcommand{\citep}[1]{\cite{#1}}
\renewcommand{\eqref}[1]{Eq~(\ref{#1})}
\renewcommand{\chapterautorefname}{Chapter}
\renewcommand{\sectionautorefname}{Section}
%% END reference structures

\begin{document}
\vspace*{0.2in}

% Title must be 250 characters or less.
\begin{flushleft}
{\Large
\textbf\newline{Adaptively temporal evolution of reproduction number estimation with trend filtering} % Please use "sentence case" for title and headings (capitalize only the first word in a title (or heading), the first word in a subtitle (or subheading), and any proper nouns).
}
\newline
% Insert author names, affiliations and corresponding author email (do not include titles, positions, or degrees).
\\
Jiaping Liu\textsuperscript{1},
Zhenglun Cai\textsuperscript{2},
Paul Gustafson\textsuperscript{1},
Daniel J. McDonald\textsuperscript{1*}
\\
\bigskip
\textbf{1} Department of Statistics, The University of British Columbia, Vancouver, British Columbia, Canada

\textbf{2} Centre for Health Evaluation and Outcome Sciences, The University of British Columbia, Vancouver, British Columbia, Canada
\\

\bigskip

% Insert additional author notes using the symbols described below. Insert symbol callouts after author names as necessary.
% 
% Remove or comment out the author notes below if they aren't used.
%
% Primary Equal Contribution Note
%\Yinyang These authors contributed equally to this work.

% Additional Equal Contribution Note
% Also use this double-dagger symbol for special authorship notes, such as senior authorship.
%\ddag These authors also contributed equally to this work.

% Current address notes
%\textcurrency Current Address: Department of Statistics, The University of British Columbia, Vancouver, British Columbia, Canada % change symbol to "\textcurrency a" if more than one current address note
% \textcurrency b Insert second current address 
% \textcurrency c Insert third current address

% Deceased author note
%\dag Deceased

% Group/Consortium Author Note
%\textpilcrow Membership list can be found in the Acknowledgments section.

% Use the asterisk to denote corresponding authorship and provide email address in note below.
* daniel@stat.ubc.ca

\end{flushleft}
% Please keep the abstract below 300 words
\section*{Abstract}
aaa

% Please keep the Author Summary between 150 and 200 words
% Use first person. PLOS ONE authors please skip this step. 
% Author Summary not valid for PLOS ONE submissions.   
\section*{Author summary}
xxx

\linenumbers

% Use "Eq" instead of "Equation" for equation citations.
\section{Introduction}
\label{sec:intro}

%% introduce Rt 
The effective reproduction number (also called the instantaneous reproduction
number) is a key quantity for understanding infectious disease dynamics
including the potential size of a pandemic, the required stringency of
prevention measures, and the efficacy of other controls. The effective
reproduction number is defined to be the average number of secondary infections
caused by a new primary infection that occurs at a specific time. Tracking the
time series of this quantity is therefore useful for understanding whether or
not future infections are likely to increase or decrease from the current state.
Let $\calR(t)$ denote the effective reproduction number at time $t$.
Practically, as long as $\calR(t) < 1$, infections will decline gradually,
eventually resulting in a disease-free equilibrium, whereas when $\calR(t) > 1$,
infections will continue to increase, resulting in endemic equilibrium. 
%% significance/necessity of modeling Rt: 
While $\calR(t)$ is fundamentally a continuous time quantity, it can be related
to data only at discrete points in time $t = 1,\ldots,n$.
This sequence of effective reproduction numbers over time is not observable, but,
nonetheless, is easily interpretable and retrospectively describes the course of
an epidemic. Therefore, a number of procedures exist to estimate $\calR_t$ from
different types of observed incidence data such as cases, deaths, or
hospitalizations, while relying on various domain-specific assumptions.
%% properties: accurate and robust to violation of assumptions 
Importantly, accurate estimation of effective reproduction numbers relies
heavily on the quality of the available data, and, due to the limitations of
data collection, such as underreporting and lack of standardization,
estimation methodologies rely on various assumptions to
compensate. Because model assumptions may not be easily verifiable from data
alone, it is also critical for any estimation procedure to be robust to model
misspecification. 

%% literature review: Bayesian approaches: EpiEstim, EpiFilter, EpiNow2, EpiNowCast
Many existing approaches for effective reproduction number estimation are
Bayesian: they estimate the posterior distribution of $\calR_t$ conditional on
the observations. One of the first such approaches is the software \EpiEstim\
\citep{cori2020package}, described by \citet{cori2013new}. This method is
prospective, in that it uses only observations available up to time $t$ in order to
estimate $\calR_t$ for each $i = 1,\ldots, t$. An advantage of \EpiEstim\ is its
straightforward statistical model: new incidence data follows the Poisson
distribution conditional on past incidence combined with the conjugate gamma prior
distribution for $\calR_t$ with fixed hyperparameters. Additionally, the serial
interval distribution is fixed and known. For this reason, \EpiEstim\ requires
little domain expertise for use, and it is computationally fast.
\citet{thompson2019improved} modified this method to distinguish imported cases from local transmission and
simultaneously estimate the serial interval distribution.
\citet{nash2023estimating} further extended \EpiEstim\ by using
``reconstructed'' daily incidence data to handle irregularly spaced observations.
% 
\cite{abbott2020estimating} proposed a Bayesian latent variable framework,
\texttt{EpiNow2} \citep{EpiNow2}, which leverages incident cases, deaths or
other available streams simultaneously along with allowing additional delay
distributions (incubation period and onset to reporting delays) in modelling.  
\citet{lison2023generative} proposed an extension that handles missing data by
imputation followed by a truncation adjustment. These modifications are intended
to increase accuracy at the most recent (but most uncertain) timepoints, to aid policymakers.
%
\citet{parag2021improved} also proposed a Bayesian approach, \texttt{EpiFilter}
based on the (discretized) Kalman filter and smoother. \texttt{EpiFilter} also
estimates the posterior of $\calR_t$ given a Gamma prior and Poisson distributed
incident cases. Compared to \EpiEstim, however, \texttt{EpiFilter} estimates
$\calR_t$ retrospectively using all available incidence data both before and
after time $t$, with the goal of being more robust in low-incidence cases.  
\citet{gressani2022epilps} proposed a Bayesian P-splines approach, \EpiLPS, that
assumes negative Binomial distributed incidence. \citet{trevisin2023spatially}
also proposed a Bayesian model estimated with particle filtering to incorporate
spatial structures.
%
Bayesian approaches estimate the posterior distribution of the effective
reproduction numbers and possess the advantage that credible intervals may be
easily computed. A limitation of many Bayesian approaches, however, is that they
usually require more intensive computational routines, especially when observed
data sequences are long or hierarchical structures are complex.  Below, we
compare our method to two of the more computationally efficient Bayesian models,
\EpiEstim\ and \EpiLPS. 

There are also frequentist approaches for $\calR_t$ estimation.
\citet{abry2020spatial} proposed regularizing the smoothness of $\calR_t$
through penalized
regression with second-order temporal regularization, additional spatial
penalties, and with Poisson loss. \citet{pascal2022nonsmooth}
extended this procedure by introducing another penalty on outliers.
%
\cite{pircalabelu2023spline} proposed a spline-based model relying on the 
assumption of exponential-family distributed incidence. 
\cite{ho2023accounting} estimates $\calR_t$ while monitoring the time-varying
level of overdispersion. 
%
There are other spline-based approaches such as
\cite{azmon2014estimation,gressani2021approximate},
autoregressive models with random effects \citep{jin2023epimix} that are robust
to low incidence cases, and generalized autoregressive moving average (GARMA)
models \citep{hettinger2023estimating} that are robust to measurement errors in
incidence data. 


%%%%%%%%%%%%%%%%%%%%%%%%%%%%%%%%%%%%%%%%%% our approach %%%%%%%%%%%%%%%%%%%%%%%%%%%%%%%%%%%%%%%%%%
We propose a retrospective effective reproduction number estimator
called \RtEstim\ that requires only incidence data. Our model makes the
conditional Poisson assumption, similar to much of the prior work described
above, but is empirically more robust to misspecification. This estimator is defined by a
convex optimization problem with Poisson loss and $\ell_1$ penalty on the
temporal evolution of $\log(\calR_t)$ to impose smoothness over time. 
As a result, \RtEstim\ generates discrete splines, and the estimated curves (in
$\log$-space) appear to be piecewise polynomials of an order selected by the
user. Importantly, The estimates are locally adaptive, meaning that different
time ranges may posses heterogeneous smoothness. Because we penalize the
logarithm of $\calR_t$, we naturally accommodate the positivity requirement, can
handle large or small incidence measurements, and are automatically (reasonably)
robust to outliers without additional constraints. A small illustration using
three years of Covid-19 case data in British Columbia is shown in \autoref{fig:intro-fig}.

\begin{figure}[tb]
    \centering
    \includegraphics[width=.99\textwidth]{fig/intro-fig-new.png}
    \caption{A demonstration of effective reproduction number estimation 
    by \RtEstim\ and the corresponding fitted incidence cases for the Covid-19 epidemic 
    in British Columbia, Canada during a period from March 28, 2020 to June 28, 2023. 
    The \textcolor{customblue}{\textbf{blue}} curve in the top panel is the estimated piecewise
    quadratic $\calR_t$ and the gray ribbon is the corresponding 95\% confidence band. 
    The black curve in the bottom panel is the observed Covid-19 daily confirmed 
    cases, and the \textcolor{customorange}{\textbf{orange}} curve is the fitted incidence 
    corresponding to the estimated $\calR_t$.}
    \label{fig:intro-fig}
\end{figure}

% Due to the limitations including but not limited to
% insufficient surveillance resources and incomplete reporting, the incidence
% cases are only observed to a certain proportion of the unknown, true values.
% Here, we assume that this proportion is smoothly time-varying, which will not
% cause an immediate changing point in the pattern of transmissibility of
% epidemics. With this assumption, we argue that the observable incidence contain
% the true curvature and changing points of the underlying effective reproduction
% numbers of epidemics. We consider a fixed serial interval distribution that
% needs to be pre-specified. It can either be chosen parameters of Gamma
% distribution or a series of probabilities provided by related methods with sum
% $1$. Serial intervals have been studied for specific infectious diseases, such
% as H1N1 influenza and Covid-19, by numerous existing studies
% \citep{white2009estimation,boelle2011transmission,rai2021estimates,alene2021serial,griffin2020rapid}. 
While our approach is straightforward and requires little domain knowledge for
implementation, we also implement a number of refinements. 
% \RtEstim\ produces accurate estimations that are empirically robust in model misspecification, i.e., the violation of distributional assumption of incidence. 
% 
 %We follow the common assumption of Poisson distributed incidence data, but the empirical study shows that the estimators are robust in negative binomial settings. 
%Thus, RtEstim is accurate, robust in model misspecification, and computationally efficient. 
% the approach we focus 
We use a proximal Newton method to solve the convex optimization problem along
with warm starts to produce estimates efficiently,
typically in a matter of seconds, even for long sequences of data.
%  Our approach takes the advantage of convex optimization and is
% solved by
% Newton's method, which is known to converge rapidly. %We provide a more
% computationally efficient algorithm for the lowest-degree problem, where
% iterative algorithms can be avoided. 
% Moreover, the sparse structure of the divided difference matrix used in the trend filtering penalty allows further efficiency in computation. 
%Temporal evolution of reproduction numbers considers the progression of reproduction numbers that occurs over time through different temporal intervals. 
In a number of simulated experiments, we show empirically that our approach is
more accurate than existing methods at estimating the true effective reproduction numbers. 


The manuscript proceeds as follows. We first introduce the methodology of
\RtEstim\ including the usage of renewal equation and the development of Poisson
trend filtering estimator. We explain how this method could be interpreted from
the Bayesian perspective, connecting it to previous work in this context. We
provide illustrative experiments comparing our estimator to \EpiEstim\ and
\EpiLPS. We then apply our \RtEstim\ on the Covid-19 pandemic incidence in
British Columbia and the 1918 influenza pandemic in the United States. Finally,
we conclude with a discussion of the advantages and limitations of our approach
and describe practical considerations for effective reproduction number
estimation.

\input{Methods}
\input{Results}
\input{Discussion}
% Results and Discussion can be combined.


\section*{Supporting information}

% Include only the SI item label in the paragraph heading. Use the \nameref{label} command to cite SI items in the text.
\paragraph*{S1 Fig.}
\label{S1_Fig}
{\bf Covid-19 figure.} Covid-19 cases in BC and the estimated reproduction numbers.

\paragraph*{S2 Fig.}
\label{S2_Fig}
{\bf Lorem ipsum.} Maecenas convallis mauris sit amet sem ultrices gravida. Etiam eget sapien nibh. Sed ac ipsum eget enim egestas ullamcorper nec euismod ligula. Curabitur fringilla pulvinar lectus consectetur pellentesque.

\paragraph*{S1 File.}
\label{S1_File}
{\bf Lorem ipsum.}  Maecenas convallis mauris sit amet sem ultrices gravida. Etiam eget sapien nibh. Sed ac ipsum eget enim egestas ullamcorper nec euismod ligula. Curabitur fringilla pulvinar lectus consectetur pellentesque.

\paragraph*{S1 Video.}
\label{S1_Video}
{\bf Lorem ipsum.}  Maecenas convallis mauris sit amet sem ultrices gravida. Etiam eget sapien nibh. Sed ac ipsum eget enim egestas ullamcorper nec euismod ligula. Curabitur fringilla pulvinar lectus consectetur pellentesque.

\paragraph*{S1 Appendix.}
\label{S1_Appendix}
{\bf Lorem ipsum.} Maecenas convallis mauris sit amet sem ultrices gravida. Etiam eget sapien nibh. Sed ac ipsum eget enim egestas ullamcorper nec euismod ligula. Curabitur fringilla pulvinar lectus consectetur pellentesque.

\paragraph*{S1 Table.}
\label{S1_Table}
{\bf Lorem ipsum.} Maecenas convallis mauris sit amet sem ultrices gravida. Etiam eget sapien nibh. Sed ac ipsum eget enim egestas ullamcorper nec euismod ligula. Curabitur fringilla pulvinar lectus consectetur pellentesque.

\section*{Acknowledgments}
Cras egestas velit mauris, eu mollis turpis pellentesque sit amet. Interdum et malesuada fames ac ante ipsum primis in faucibus. Nam id pretium nisi. Sed ac quam id nisi malesuada congue. Sed interdum aliquet augue, at pellentesque quam rhoncus vitae.

\nolinenumbers

% Either type in your references using
% \begin{thebibliography}{}
% \bibitem{}
% Text
% \end{thebibliography}
%
% or
%
% Compile your BiBTeX database using our plos2015.bst
% style file and paste the contents of your .bbl file
% here. See http://journals.plos.org/plosone/s/latex for 
% step-by-step instructions.
% 
%\begin{thebibliography}{10}
%
%  \bibitem{diekmann1990definition}
%  Diekmann O, Heesterbeek JAP, Metz JA.
%  \newblock On the definition and the computation of the basic reproduction ratio
%    R 0 in models for infectious diseases in heterogeneous populations.
%  \newblock Journal of mathematical biology. 1990;28:365--382.
%  
%  \bibitem{dietz1993estimation}
%  Dietz K.
%  \newblock The estimation of the basic reproduction number for infectious
%    diseases.
%  \newblock Statistical methods in medical research. 1993;2(1):23--41.
%  
%  \bibitem{fine1993herd}
%  Fine PE.
%  \newblock Herd immunity: history, theory, practice.
%  \newblock Epidemiologic reviews. 1993;15(2):265--302.
%  
%  \bibitem{delamater2019complexity}
%  Delamater PL, Street EJ, Leslie TF, Yang YT, Jacobsen KH.
%  \newblock Complexity of the basic reproduction number (R0).
%  \newblock Emerging infectious diseases. 2019;25(1):1.
%  
%  \bibitem{brauer2019mathematical}
%  Brauer F, Castillo-Chavez C, Feng Z.
%  \newblock Mathematical models in epidemiology. vol.~32.
%  \newblock Springer; 2019.
%  
%  \bibitem{anderson1982directly}
%  Anderson RM, May RM.
%  \newblock Directly transmitted infections diseases: control by vaccination.
%  \newblock Science. 1982;215(4536):1053--1060.
%  
%  \bibitem{anderson1985vaccination}
%  Anderson RM, May RM.
%  \newblock Vaccination and herd immunity to infectious diseases.
%  \newblock Nature. 1985;318(6044):323--329.
%  
%  \bibitem{heffernan2005perspectives}
%  Heffernan JM, Smith RJ, Wahl LM.
%  \newblock Perspectives on the basic reproductive ratio.
%  \newblock Journal of the Royal Society Interface. 2005;2(4):281--293.
%  
%  \bibitem{cori2013new}
%  Cori A, Ferguson NM, Fraser C, Cauchemez S.
%  \newblock A new framework and software to estimate time-varying reproduction
%    numbers during epidemics.
%  \newblock American journal of epidemiology. 2013;178(9):1505--1512.
%  
%  \bibitem{abry2020spatial}
%  Abry P, Pustelnik N, Roux S, Jensen P, Flandrin P, Gribonval R, et~al.
%  \newblock Spatial and temporal regularization to estimate COVID-19 reproduction
%    number R (t): Promoting piecewise smoothness via convex optimization.
%  \newblock Plos one. 2020;15(8):e0237901.
%  
%  \bibitem{pascal2022nonsmooth}
%  Pascal B, Abry P, Pustelnik N, Roux S, Gribonval R, Flandrin P.
%  \newblock Nonsmooth convex optimization to estimate the Covid-19 reproduction
%    number space-time evolution with robustness against low quality data.
%  \newblock IEEE Transactions on Signal Processing. 2022;70:2859--2868.
%  
%  \bibitem{thompson2019improved}
%  Thompson RN, Stockwin JE, van Gaalen RD, Polonsky JA, Kamvar ZN, Demarsh PA,
%    et~al.
%  \newblock Improved inference of time-varying reproduction numbers during
%    infectious disease outbreaks.
%  \newblock Epidemics. 2019;29:100356.
%  
%  \bibitem{ramdas2016fast}
%  Ramdas A, Tibshirani RJ.
%  \newblock Fast and flexible admm algorithms for trend filtering.
%  \newblock Journal of Computational and Graphical Statistics.
%    2016;25(3):839--858.
%  
%\end{thebibliography}
  
\bibliography{ptf}
\end{document}
