\documentclass[12pt]{article}
% header.tex

% page layout
%\usepackage[left=1in,right=1in,top=1in,footskip=25pt]{geometry} 
\usepackage{setspace}
%\doublespacing
\onehalfspacing
% \usepackage{changepage}

% define math norms
\usepackage{amsmath,amsthm,amssymb,amsfonts}
\usepackage{mathtools}
%\DeclarePairedDelimiter\norm\lVert\rVert

%% define a rotated tilted `less & greater than' with certain rotated angles
\usepackage{stackengine,graphicx,scalerel}
\newcommand\rotlesssim{\mathrel{\ensurestackMath{\ThisStyle{%
  \stackengine{-.4\LMex}{\SavedStyle<}{%
    \rotatebox{-15}{$\SavedStyle\sim$}}{U}{r}{F}{T}{S}}}}}
\newcommand\rotgtrsim{\mathrel{\ensurestackMath{\ThisStyle{%
  \stackengine{-.4\LMex}{\SavedStyle>}{%
    \rotatebox{15}{$\SavedStyle\sim$}}{U}{l}{F}{T}{S}}}}}

% notations
\usepackage{etoolbox}
\usepackage{datagidx}

% lists
\usepackage{enumitem}
% Load dsfont this to get proper indicator function (bold 1) with \mathds{1}:
\usepackage{dsfont}
\usepackage{centernot}

% graphics
\DeclareGraphicsExtensions{.pdf,.png,.jpg}
\graphicspath{ {fig/} }
%\usepackage{wrapfig}
%\usepackage{graphics}

% tables
\usepackage{multirow}
%\usepackage{float}
%\setlength{\footskip}{40pt}
%\font\myfont=cmr12 at 20pt

% theory
\newtheorem{lemma}{Lemma}
\newtheorem{theorem}{Theorem}

% algorithms
\usepackage{algorithm2e}%[linesnumbered,ruled,vlined]

% table of contents
\usepackage[utf8]{inputenc}
%\usepackage{blindtext}

% others: appendix, color and more
\usepackage{xcolor}
%\usepackage{appendix}
\usepackage{titlesec}

%\usepackage{caption}
%\PassOptionsToPackage{numbers,square,sort&compress}{natbib}
%\bibliographystyle{plainnat}
\usepackage{fancyhdr}
\pagestyle{fancy}
\usepackage[fit]{truncate}
\fancyhead[LE,RO]{\nouppercase{\truncate{0.5\headwidth}{\rightmark}}}
\fancyhead[LO,RE]{\nouppercase{\truncate{0.5\headwidth}{\leftmark}}}

%\fancyhf{}
%\renewcommand{\thepage}{\roman{page}}
\pagenumbering{roman}
%\setlength{\headheight}{14.5pt}

% references
%\usepackage[authoryear,round,sort&compress]{natbib}%,round,compress,elide,comma
%\bibliographystyle{plainnat}
%\usepackage[american]{babel}


% hyperref should come at the bottom
% Set up hyperlinks:
\definecolor{RefColor}{rgb}{0,0,.65}
\usepackage[colorlinks,linkcolor=RefColor,citecolor=RefColor,urlcolor=RefColor]{hyperref} %hidelinks
\renewcommand{\eqref}[1]{Equation~(\ref{#1})}
\renewcommand{\chapterautorefname}{Chapter}
\renewcommand{\sectionautorefname}{Section}

%\usepackage[capitalize]{cleveref}
%\crefname{appsec}{appendix}{appendices}
%\Crefname{appsec}{Appendix}{Appendices}

\newcommand{\attn}[1]{\textcolor{red}{ATTN: #1}}
\renewcommand{\hat}{\widehat}

% defs.tex
% define custom notation, commands, etc.

%%%%%%%%%%%%%%%%%%%%%%%%%%%%%%%%%%%%%%%%%%%%%%%%%%%%%%%%%%
%% full alphabets of different styles 
% bf series
\def\bfA{\mathbf{A}}
\def\bfB{\mathbf{B}}
\def\bfC{\mathbf{C}}
\def\bfD{\mathbf{D}}
\def\bfE{\mathbf{E}}
\def\bfF{\mathbf{F}}
\def\bfG{\mathbf{G}}
\def\bfH{\mathbf{H}}
\def\bfI{\mathbf{I}}
\def\bfJ{\mathbf{J}}
\def\bfK{\mathbf{K}}
\def\bfL{\mathbf{L}}
\def\bfM{\mathbf{M}}
\def\bfN{\mathbf{N}}
\def\bfO{\mathbf{O}}
\def\bfP{\mathbf{P}}
\def\bfQ{\mathbf{Q}}
\def\bfR{\mathbf{R}}
\def\bfS{\mathbf{S}}
\def\bfT{\mathbf{T}}
\def\bfU{\mathbf{U}}
\def\bfV{\mathbf{V}}
\def\bfW{\mathbf{W}}
\def\bfX{\mathbf{X}}
\def\bfY{\mathbf{Y}}
\def\bfZ{\mathbf{Z}}

% bb series
\def\bbA{\mathbb{A}}
\def\bbB{\mathbb{B}}
\def\bbC{\mathbb{C}}
\def\bbD{\mathbb{D}}
\def\bbE{\mathbb{E}}
\def\bbF{\mathbb{F}}
\def\bbG{\mathbb{G}}
\def\bbH{\mathbb{H}}
\def\bbI{\mathbb{I}}
\def\bbJ{\mathbb{J}}
\def\bbK{\mathbb{K}}
\def\bbL{\mathbb{L}}
\def\bbM{\mathbb{M}}
\def\bbN{\mathbb{N}}
\def\bbO{\mathbb{O}}
\def\bbP{\mathbb{P}}
\def\bbQ{\mathbb{Q}}
\def\bbR{\mathbb{R}}
\def\bbS{\mathbb{S}}
\def\bbT{\mathbb{T}}
\def\bbU{\mathbb{U}}
\def\bbV{\mathbb{V}}
\def\bbW{\mathbb{W}}
\def\bbX{\mathbb{X}}
\def\bbY{\mathbb{Y}}
\def\bbZ{\mathbb{Z}}

% cal series
\def\calA{\mathcal{A}}
\def\calB{\mathcal{B}}
\def\calC{\mathcal{C}}
\def\calD{\mathcal{D}}
\def\calE{\mathcal{E}}
\def\calF{\mathcal{F}}
\def\calG{\mathcal{G}}
\def\calH{\mathcal{H}}
\def\calI{\mathcal{I}}
\def\calJ{\mathcal{J}}
\def\calK{\mathcal{K}}
\def\calL{\mathcal{L}}
\def\calM{\mathcal{M}}
\def\calN{\mathcal{N}}
\def\calO{\mathcal{O}}
\def\calP{\mathcal{P}}
\def\calQ{\mathcal{Q}}
\def\calR{\mathcal{R}}
\def\calS{\mathcal{S}}
\def\calT{\mathcal{T}}
\def\calU{\mathcal{U}}
\def\calV{\mathcal{V}}
\def\calW{\mathcal{W}}
\def\calX{\mathcal{X}}
\def\calY{\mathcal{Y}}
\def\calZ{\mathcal{Z}}
%%%%%%%%%%%%%%%%%%%%%%%%%%%%%%%%%%%%%%%%%%%%%%%%%%%%%%%%%%

%%%%%%%%%%%%%%%%%%%%%%%%%%%%%%%%%%%%%%%%%%%%%%%%%%%%%%%%%%
% quasi-universal probabilistic and mathematical notation
% my preferences (modulo publication conventions, and clashes like random vectors):
%   vectors: bold, lowercase
%   matrices: bold, uppercase
%   operators: blackboard (e.g., \mathbb{E}), uppercase
%   sets, spaces: calligraphic, uppercase
%   random variables: normal font, uppercase
%   deterministic quantities: normal font, lowercase

\newcommand{\R}{\texttt{R}\ }
\newcommand{\cpp}{\texttt{C++}\ }

% lists
\makeatletter
\newcommand{\blist}[1]{\begin{enumerate}[label=\roman*.]\label{list:#1}}
\newcommand{\elist}{\end{enumerate}}
\makeatother

% equations
\makeatletter
\newcommand{\beq}[1]{\begin{equation}\label{eq:#1}}
\newcommand{\eeq}{\end{equation}}
\makeatother

\newcommand{\fr}[1]{\frac{1}{#1}}
\newcommand{\lr}[1]{\left(#1\right)}
\newcommand{\norm}[1]{\left\lVert #1 \right\rVert}
\newcommand{\abs}[1]{\left\lvert #1 \right\rvert}

\DeclareMathOperator*{\Lambert}{\mathrm{Lambert}_0}
\DeclareMathOperator*{\diag}{diag}
\DeclareMathOperator*{\argmin}{argmin}
\newcommand{\Argmin}[1]{\underset{#1}{\argmin\ }}
\DeclareMathOperator*{\argmax}{argmax}
\newcommand{\Argmax}[1]{\underset{#1}{\argmax\ }}
\def\sumN{\sum_{i=1}^n}
\def\RtEstim{\texttt{RtEstim}}
\def\EpiEstim{\texttt{EpiEstim}}
\def\EpiLPS{\texttt{EpiLPS}}

% binary relations
\def\condind{{\perp\!\!\!\perp}} %independence/conditional independence
\def\equdist{\stackrel{\text{\rm\tiny d}}{=}} %equal in distribution
\def\equas{\stackrel{\text{\rm\tiny a.s.}}{=}} %equal almost surely
\def\simiid{\sim_{\mbox{\tiny iid}}} %sampled i.i.d

% common vectors and matrices
\def\onevec{\mathbf{1}}
\def\iden{\mathbf{I}} % identity matrix
\def\supp{\text{\rm supp}}

\def\gDeltakk{\Delta^{(k+1)}}
\def\gDeltakkX{\Delta^{(x,k+1)}}
\def\gDeltak{\Delta^{(k)}}
\def\gDeltakX{\Delta^{(x,k)}}
\def\gDelta{\Delta^{(1)}}
\def\gDeltaX{\Delta^{(x,1)}}
\def\gDkk{D^{(k+1)}}
\def\gDk{D^{(k)}}
\def\gD1{D^{(1)}}
\def\Dxk{D^{(x,k)}}
\def\Dxkk{D^{(x,k+1)}}
\def\tDxk{\tilde{D}^{(x,k+1)}}

% misc
% floor and ceiling
\DeclarePairedDelimiter{\ceilpair}{\lceil}{\rceil}
\DeclarePairedDelimiter{\floor}{\lfloor}{\rfloor}
\newcommand{\argdot}{{\,\vcenter{\hbox{\tiny$\bullet$}}\,}} %generic argument dot
%%%%%%%%%%%%%%%%%%%%%%%%%%%%%%%%%%%%%%%%%%%%%%%%%%%%%%%%%%

%%%%%%%%%%%%%%%%%%%%%%%%%%%%%%%%%%%%%%%%%%%%%%%%%%%%%%%%%%
% Notations 

% Define a database of notations
\newgidx{notation}{Notation}
% Set this as the default database so we don't have to keep
% explicitly mentioning it
\DTLgidxSetDefaultDB{notation}

% Define some terms
\newterm[description={description for A}]{\gDeltakk}

% Fake a counter called "notation" to be used in the location list
% Two digits are used to make the substring comparison easier
% (used in the condition of \printterms)
%\makeatletter
%\newcommand*{\twodigits}[1]{\two@digits{\value{#1}}}
%\newcommand*{\thenotation}{C\twodigits{chapter}.\thepage}
%\makeatother


% Define a command to print the notation list
\newcommand{\printnotation}{
    \bgroup
    \edef\locationprefix{}
    \printterms
     [heading={\section*}, % put the heading in an unnumbered section    
     condition={\DTLisSubString{\Location}{\locationprefix}},
     sort={}, % suppress sorting - do it manually    
     location=first, % only show first location    
     columns=1 % one column list - change as required   
     ]%
    \egroup
    % Add a vertical gap so it's not right on top of the next paragraph
    \vskip\baselineskip
    % Set the after heading flag (remove if not required):
    \csuse{@afterheading}
    % Something is confusing the rerun warnings, so disable it and 
    % just remember to use multiple LaTeX runs: 
    \csdef{@datagidx@dorerun@warn}{}%
}

% Patch \datagidx@formatlocation so that it strips the location prefix
% and inserts "p." before the location
\makeatletter
\let\orgformatlocation\datagidx@formatlocation
\renewcommand*{\datagidx@formatlocation}[2]{%
  \expandafter\DTLsplitstring\expandafter{#2}{.}{\locprefix}{\locsuffix}%
  \orgformatlocation{#1}{p.\locsuffix}%
}
\makeatother

% after all terms have been defined, so manually to avoid the
% redundancy of sorting each time the notation list is display.
\dtlsort{Sort}{notation}{\dtlwordindexcompare}


\usepackage[utf8]{inputenc} % allow utf-8 input
\usepackage[T1]{fontenc}    % use 8-bit T1 fonts
\usepackage{url}            % simple URL typesetting
\usepackage{booktabs}       % professional-quality tables
\usepackage{amsfonts}       % blackboard math symbols
\usepackage{nicefrac}       % compact symbols for 1/2, etc.
\usepackage{microtype}      % microtypography
\usepackage{lipsum}
% \usepackage{graphicx}
\graphicspath{ {./images/} }
\usepackage[margin=2.5cm,nohead]{geometry}
\usepackage{authblk}
\newcommand{\email}[1]{\href{mailto:#1}{#1}}

\title{RtEstim: Effective reproduction number estimation with trend filtering}

\author[a,1]{Jiaping Liu}
\author[b]{Zhenglun Cai}
\author[a]{Paul Gustafson}
\author[a]{Daniel J. McDonald}

\affil[a]{Department of Statistics, The University of British Columbia}
\affil[b]{Centre for Health Evaluation and Outcome Sciences, The University of
British Columbia}

\begin{document}

\maketitle

\begin{abstract}
  To understand the transmissibility and spread of infectious diseases,
  epidemiologists turn to estimates of the effective reproduction number.
  While many approaches exist, their utility may be limited by the
  challenges of surveillance data collection. Arbitrary model assumptions
  that are unverifiable with data alone and sophisticated, though
  computationally inefficient, frameworks that
  are critical limitations for many
  existing approaches. We propose a discrete spline-based approach 
  \RtEstim\ that solves a convex optimization problem --- Poisson trend filtering
  --- using the proximal Newton method. 
  It produces a locally adaptive estimator for effective reproduction number 
  estimation with heterogeneous smoothness. 
  \RtEstim\ remains accurate even under some process misspecifications and
  is computationally efficient, even for large-scale data. The 
  implementation is easily accessible in a lightweight \texttt{R} package
  \href{https://dajmcdon.github.io/rtestim/index.html}{\texttt{rtestim}}.

  {\bf Keywords:} tuning parameter selection $|$ heterogeneous smoothness
\end{abstract}

\renewcommand{\abstractname}{Author summary}
\begin{abstract}
  Effective reproduction number estimation faces many challenges, for example, 
  data is usually collected from multiple sources and there are reporting delays, that cause many problems, 
  such as missing values, underreporting and lack of standardization. 
  Such poor conditions of data hinder the accurate estimation of effective reproduction number.  
  Our motivation is to develop a model that returns accurate estimation, which is robust to violation of model misspecification, and a tool that is straightforward to use and computationally efficient, even for large-scale data. 
  Considering the temporal evolution of effective reproduction number for multiple degrees, we propose a convex optimization model with an $\ell_1$ trend filtering penalty for univariate data. We solve our model using the proximal Newton method, which converges rapidly and is numerically stable. 
  Our software, \texttt{R} package \RtEstim, can return the output in seconds for univariate incidence sequences with length 300 using candidate sets of length fifty and ten-fold cross validation. 
\end{abstract}

\footnotetext[1]{To whom correspondence should be
  addressed. E-mail: \email{jiaping.liu@stat.ubc.ca}} 
% Technical summary including research objective, method, results, and conclusions.

%\thispagestyle{empty}
%\pagenumbering{roman}
%\setcounter{page}{1}

%\printnotation
%\input{doc/notations.tex}
%\newpage
%\hypersetup{linkcolor=black}
\tableofcontents % for easy edits, remove later

%\listoffigures

\clearpage
%\pagenumbering{arabic}
%\setcounter{page}{1}

% research proposal
\section{Introduction}
\label{sec:intro}

%% introduce Rt 
The effective reproduction number (also called the instantaneous reproduction
number) is a key quantity for understanding infectious disease dynamics
including the potential size of a pandemic, the required stringency of
prevention measures, and the efficacy of other controls. The effective
reproduction number is defined to be the average number of secondary infections
caused by a new primary infection that occurs at a specific time. Tracking the
time series of this quantity is therefore useful for understanding whether or
not future infections are likely to increase or decrease from the current state.
Let $\calR(t)$ denote the effective reproduction number at time $t$.
Practically, as long as $\calR(t) < 1$, infections will decline gradually,
eventually resulting in a disease-free equilibrium, whereas when $\calR(t) > 1$,
infections will continue to increase, resulting in endemic equilibrium. 
%% significance/necessity of modeling Rt: 
While $\calR(t)$ is fundamentally a continuous time quantity, it can be related
to data only at discrete points in time $t = 1,\ldots,n$.
This sequence of effective reproduction numbers over time is not observable, but,
nonetheless, is easily interpretable and retrospectively describes the course of
an epidemic. Therefore, a number of procedures exist to estimate $\calR_t$ from
different types of observed incidence data such as cases, deaths, or
hospitalizations, while relying on various domain-specific assumptions.
%% properties: accurate and robust to violation of assumptions 
Importantly, accurate estimation of effective reproduction numbers relies
heavily on the quality of the available data, and, due to the limitations of
data collection, such as underreporting and lack of standardization,
estimation methodologies rely on various assumptions to
compensate. Because model assumptions may not be easily verifiable from data
alone, it is also critical for any estimation procedure to be robust to model
misspecification. 

%% literature review: Bayesian approaches: EpiEstim, EpiFilter, EpiNow2, EpiNowCast
Many existing approaches for effective reproduction number estimation are
Bayesian: they estimate the posterior distribution of $\calR_t$ conditional on
the observations. One of the first such approaches is the software \EpiEstim\
\citep{cori2020package}, described by \citet{cori2013new}. This method is
prospective, in that it uses only observations available up to time $t$ in order to
estimate $\calR_t$ for each $i = 1,\ldots, t$. An advantage of \EpiEstim\ is its
straightforward statistical model: new incidence data follows the Poisson
distribution conditional on past incidence combined with the conjugate gamma prior
distribution for $\calR_t$ with fixed hyperparameters. Additionally, the serial
interval distribution is fixed and known. For this reason, \EpiEstim\ requires
little domain expertise for use, and it is computationally fast.
\citet{thompson2019improved} modified this method to distinguish imported cases from local transmission and
simultaneously estimate the serial interval distribution.
\citet{nash2023estimating} further extended \EpiEstim\ by using
``reconstructed'' daily incidence data to handle irregularly spaced observations.
% 
\cite{abbott2020estimating} proposed a Bayesian latent variable framework,
\texttt{EpiNow2} \citep{EpiNow2}, which leverages incident cases, deaths or
other available streams simultaneously along with allowing additional delay
distributions (incubation period and onset to reporting delays) in modelling.  
\citet{lison2023generative} proposed an extension that handles missing data by
imputation followed by a truncation adjustment. These modifications are intended
to increase accuracy at the most recent (but most uncertain) timepoints, to aid policymakers.
%
\citet{parag2021improved} also proposed a Bayesian approach, \texttt{EpiFilter}
based on the (discretized) Kalman filter and smoother. \texttt{EpiFilter} also
estimates the posterior of $\calR_t$ given a Gamma prior and Poisson distributed
incident cases. Compared to \EpiEstim, however, \texttt{EpiFilter} estimates
$\calR_t$ retrospectively using all available incidence data both before and
after time $t$, with the goal of being more robust in low-incidence cases.  
\citet{gressani2022epilps} proposed a Bayesian P-splines approach, \EpiLPS, that
assumes negative Binomial distributed incidence. \citet{trevisin2023spatially}
also proposed a Bayesian model estimated with particle filtering to incorporate
spatial structures.
%
Bayesian approaches estimate the posterior distribution of the effective
reproduction numbers and possess the advantage that credible intervals may be
easily computed. A limitation of many Bayesian approaches, however, is that they
usually require more intensive computational routines, especially when observed
data sequences are long or hierarchical structures are complex.  Below, we
compare our method to two of the more computationally efficient Bayesian models,
\EpiEstim\ and \EpiLPS. 

There are also frequentist approaches for $\calR_t$ estimation.
\citet{abry2020spatial} proposed regularizing the smoothness of $\calR_t$
through penalized
regression with second-order temporal regularization, additional spatial
penalties, and with Poisson loss. \citet{pascal2022nonsmooth}
extended this procedure by introducing another penalty on outliers.
%
\cite{pircalabelu2023spline} proposed a spline-based model relying on the 
assumption of exponential-family distributed incidence. 
\cite{ho2023accounting} estimates $\calR_t$ while monitoring the time-varying
level of overdispersion. 
%
There are other spline-based approaches such as
\cite{azmon2014estimation,gressani2021approximate},
autoregressive models with random effects \citep{jin2023epimix} that are robust
to low incidence cases, and generalized autoregressive moving average (GARMA)
models \citep{hettinger2023estimating} that are robust to measurement errors in
incidence data. 


%%%%%%%%%%%%%%%%%%%%%%%%%%%%%%%%%%%%%%%%%% our approach %%%%%%%%%%%%%%%%%%%%%%%%%%%%%%%%%%%%%%%%%%
We propose a retrospective effective reproduction number estimator
called \RtEstim\ that requires only incidence data. Our model makes the
conditional Poisson assumption, similar to much of the prior work described
above, but is empirically more robust to misspecification. This estimator is defined by a
convex optimization problem with Poisson loss and $\ell_1$ penalty on the
temporal evolution of $\log(\calR_t)$ to impose smoothness over time. 
As a result, \RtEstim\ generates discrete splines, and the estimated curves (in
$\log$-space) appear to be piecewise polynomials of an order selected by the
user. Importantly, The estimates are locally adaptive, meaning that different
time ranges may posses heterogeneous smoothness. Because we penalize the
logarithm of $\calR_t$, we naturally accommodate the positivity requirement, can
handle large or small incidence measurements, and are automatically (reasonably)
robust to outliers without additional constraints. A small illustration using
three years of Covid-19 case data in British Columbia is shown in \autoref{fig:intro-fig}.

\begin{figure}[tb]
    \centering
    \includegraphics[width=.99\textwidth]{fig/intro-fig-new.png}
    \caption{A demonstration of effective reproduction number estimation 
    by \RtEstim\ and the corresponding fitted incidence cases for the Covid-19 epidemic 
    in British Columbia, Canada during a period from March 28, 2020 to June 28, 2023. 
    The \textcolor{customblue}{\textbf{blue}} curve in the top panel is the estimated piecewise
    quadratic $\calR_t$ and the gray ribbon is the corresponding 95\% confidence band. 
    The black curve in the bottom panel is the observed Covid-19 daily confirmed 
    cases, and the \textcolor{customorange}{\textbf{orange}} curve is the fitted incidence 
    corresponding to the estimated $\calR_t$.}
    \label{fig:intro-fig}
\end{figure}

% Due to the limitations including but not limited to
% insufficient surveillance resources and incomplete reporting, the incidence
% cases are only observed to a certain proportion of the unknown, true values.
% Here, we assume that this proportion is smoothly time-varying, which will not
% cause an immediate changing point in the pattern of transmissibility of
% epidemics. With this assumption, we argue that the observable incidence contain
% the true curvature and changing points of the underlying effective reproduction
% numbers of epidemics. We consider a fixed serial interval distribution that
% needs to be pre-specified. It can either be chosen parameters of Gamma
% distribution or a series of probabilities provided by related methods with sum
% $1$. Serial intervals have been studied for specific infectious diseases, such
% as H1N1 influenza and Covid-19, by numerous existing studies
% \citep{white2009estimation,boelle2011transmission,rai2021estimates,alene2021serial,griffin2020rapid}. 
While our approach is straightforward and requires little domain knowledge for
implementation, we also implement a number of refinements. 
% \RtEstim\ produces accurate estimations that are empirically robust in model misspecification, i.e., the violation of distributional assumption of incidence. 
% 
 %We follow the common assumption of Poisson distributed incidence data, but the empirical study shows that the estimators are robust in negative binomial settings. 
%Thus, RtEstim is accurate, robust in model misspecification, and computationally efficient. 
% the approach we focus 
We use a proximal Newton method to solve the convex optimization problem along
with warm starts to produce estimates efficiently,
typically in a matter of seconds, even for long sequences of data.
%  Our approach takes the advantage of convex optimization and is
% solved by
% Newton's method, which is known to converge rapidly. %We provide a more
% computationally efficient algorithm for the lowest-degree problem, where
% iterative algorithms can be avoided. 
% Moreover, the sparse structure of the divided difference matrix used in the trend filtering penalty allows further efficiency in computation. 
%Temporal evolution of reproduction numbers considers the progression of reproduction numbers that occurs over time through different temporal intervals. 
In a number of simulated experiments, we show empirically that our approach is
more accurate than existing methods at estimating the true effective reproduction numbers. 


The manuscript proceeds as follows. We first introduce the methodology of
\RtEstim\ including the usage of renewal equation and the development of Poisson
trend filtering estimator. We explain how this method could be interpreted from
the Bayesian perspective, connecting it to previous work in this context. We
provide illustrative experiments comparing our estimator to \EpiEstim\ and
\EpiLPS. We then apply our \RtEstim\ on the Covid-19 pandemic incidence in
British Columbia and the 1918 influenza pandemic in the United States. Finally,
we conclude with a discussion of the advantages and limitations of our approach
and describe practical considerations for effective reproduction number
estimation.
  
\section{Methods}

\subsection{Renewal model for incidence data} 

The effective reproduction number $\calR(t)$
is defined to be the expected number of secondary infections at time $t$
produced by a primary infection sometime in the past.
To make this precise, denote the number
of new infections at time $t$ as $y_t$. Then the total primary
infectiousness can be written as $\lambda(t) := \int_0^{\infty} p(i) y(t-i)
\diff i$,
where $p(i)$ is the probability that a new secondary infection is the result
primary infection which occurred $i$ time units in the past. 
The reproduction number is
then given as the value that equates
\begin{equation} \label{eq:pre-renew-equation}
  y(t) = \calR(t)\lambda(t) = \calR(t)\int_0^\infty p(i)y(t-i)\diff i,
\end{equation}
otherwise known as the renewal equation. 
The period between primary and secondary
infections is exactly the generation time of the disease, but given real data,
observed at discrete times (say, daily) this delay distribution must be discretized
into contiguous time intervals,
that is, $(0,1], (1,2], \dots$ resulting in the sequence $\{p_i\}_1^\infty$
corresponding to observations $y_t$ and resulting in the
discretized version of \eqref{eq:pre-renew-equation},
\begin{equation} \label{eq:renew-equation}
  y_t = \calR_t\lambda_t = \calR_t\sum_{i = 0}^\infty p_i y_{t-i}.
\end{equation}
Many approaches to estimating $\calR_t$ rely on \eqref{eq:renew-equation} as
motivation for their procedures, among them,
\EpiEstim\ \citep{cori2013new} and \texttt{EpiFilter}
\citep{parag2021improved}. 

% serial interval probabilities
In most cases, it is safe to assume that
infectiousness disappears beyond $\tau$ timepoints ($p(i) = 0$ for $i > \tau$)
so that the truncated integral of the generation interval distribution
$\int_0^\tau p(i)\diff i = 1$. 
Generation time, however, is usually unobservable and tricky to estimate, so
common practice is to approximate it by the serial interval: the period between
the symptom onsets of primary and secondary infections. If the infectiousness
profile after symptom onset is independent of the incubation period (the period
from the time of infection to the time of symptom onset), then this
approximation is justifiable: the serial interval distribution and the
generation interval distribution share the same mean. However, other properties
may not be similarly shared, and, in general, the generation interval
distribution is a convolution of the serial interval distribution with the
distribution of the differance between independent draws from the delay
distribution from infection to symptom onset. See, for example,
\citep{gostic2020practical} for a fuller discussion of the dangers of this
approximation. Nonetheless, treating these as interchangable is common
\citep{cori2013new} and beyond the scope of this work. Additionally, we assume
that the generation interval (and, therefore, the serial interval), is constant over time
$t$. That is, the probability $p(i)$ depends only on the gap between primary and
secondary infections and not on the time $t$ when the secondary infection
occurs. For our methods, we will assume that the serial interval can be
accurately estimated from auxilliary data (say by contact tracing, or previous
epidemics) and we will take it as fixed, as is common in existing studies, e.g.,
\cite{cori2013new,abry2020spatial,pascal2022nonsmooth}.

% advantage of using the renewal equation
The renewal equation in \eqref{eq:renew-equation} relates observable data
streams (incident cases) occurring at different time points to the reproduction
number given the serial interval. The fact that it depends only on the observed
incidence counts makes it reasonable to estimate $\calR_t$. However, this
relationship obscures some difficulties in data collection. Diagnostic testing
targets symptomatic individuals, omitting asymptomatic primary infections which
can lead to future secondary infections. Testing practices, availability, and
uptake can vary across space and time \citep{pitzer2021impact,
hitchings2021usefulness}. Finally, incident cases as reported to public health
are subject to delays due to laboratory confirmation, test turnaround times, and
eventual submission to public health \citep{pellis2021challenges}. For these
reasons, reported cases are lagging indicators of the course of the pandemic.
Furthermore, they do not represent the actual number of new infections that
occur on a given day, as indicated by exposure to the pathogen. The assumptions
described above (constant serial interval distribution, homogenous mixing,
similar susceptibility and social behaviours, etc.) are therefore consequential.
That said, \eqref{eq:renew-equation} also provides some comfort about deviations
from these assumptions. If $y_t$ is scaled by a constant (in time) describing
the reporting ratio, then it will cancel from both sides. Similar arguments mean
that even if such a scaling varies in time, as long as it varies slowly relative
to the set of $p_i$ that are larger than 0, \eqref{eq:renew-equation} will be a
reasonably accurate approximation, so that $\calR_t$ can still be estimated well
from reported incidence data. Finally, even a sudden change, say from $c_1$ for
$i=1,\ldots,t_1$ to $c_2$ for $i>t_1$ would only result in large errors for $t$
in the neighbourhood of $t_1$ (where the size of this neighbourhood is again
determined by the effective support of $\{p_i\}$). This robustness to certain
types of data reporting issues provides some degree of comfort when depending on
\eqref{eq:renew-equation} to calculate $\calR_t$.

\subsection{Poisson trend filtering estimator} %Proximal optimization

We use the daily confirmed incident cases $y_t$ on day $t$ to estimate the
observed infectious cases under the model that $y_t$ given previous incident
cases $y_{t-1},\ldots,y_1$ and a constant serial interval distribution follows a
Poisson distribution with mean $\lambda_t$. That is, $y_t \sim
\mathrm{Poisson}(\lambda_t)$, where $\lambda_t =  \calR_t\sum_{i=0}^{t}p_i
y_{t-i}$. Given a history of $n$ confirmed incidence counts $\bfy = (y_1,\ldots,y_n)^\top$,
our interest is to estimate $\calR_t$. A natural approach is to maximize the
likelihood, producing the MLE:
\begin{equation} \label{eq:mle}
  \begin{split}
    \widehat{\calR} &= \Argmax{\calR \in \bbR_+^n} \bbP(\calR \mid \bfy,\ \bfp)
    = \Argmax{\calR \in \bbR^n_+} \prod_{t = 1,\dots,n} 
    \frac{e^{- \calR_t \lambda_t} \lr{\calR_t \lambda_t}^{y_t} }{y_t!}\\
    &= \Argmin{\calR\in\bbR^n_+} \frac{1}{n}\sum_{i = 1}^n \calR_t\lambda_t - y_t\log(\calR_t\lambda_t).
  \end{split}
\end{equation}
This optimization problem, however, is easily seen to yields a one-to-one
correspondence between the confirmed cases and the effective reproduction, i.e.
$\widehat{\calR}_t = y_t / \lambda_t$, so that the estimated sequence
$\widehat{\calR}$ will have no significant graphical smoothness.

The MLE provides an unbiased estimation of the true observations, but leads to
high variance at the same time. We introduce smoothness of the effective
reproduction numbers into the model to decrease the variance, leading to more
accurate estimation (assuming that neighbouring values of $\calR_t$ are
similar to each other).
Meanwhile, the smoothed estimation can still keep the critical changing points
of the transmissibility for the reference to policy makers. 
%Smoothness of the effective reproduction numbers is a key to understand the trend of transmissibility of infectious diseases in retrospective studies. 
Smoother estimated curves give more high-level ideas with less changing points
and hide minor details, and vice versa. We assume the effective reproduction
numbers to appear as piecewise polynomials with multiple knots (i.e., changing
points of graphical curvature) with varying degrees. We specifically consider
discrete splines with various degrees of continuity. For instance, the $0$th
degree discrete splines are piecewise constant, the $1$st degree curves are
piecewise linear, and the $2$nd degree curves are piecewise quadratic. For
$k\geq 1$, the $k$th degree discrete splines are continuous and have continuous
discrete differences up to degree $k-1$ at the knots. 

To achieve such smoothness, we regularize the distance between adjacent
effective reproduction numbers. Since $\calR_t > 0$, penalizing the distance
between $\calR_t$s directly may cause numerical issues such that there may be
negative estimates generated in computation. Therefore, we equivalently penalize
the distance between natural logarithms of neighboring $\calR_t$s through
divided differences (i.e., discrete derivatives) with various orders.  
Compared to splines, discrete splines introduce computational efficiency without
loss of numerical accuracy. We penalize $\ell_1$ norm of the distance, which
introduces sparsity into the curvature, so that the estimates have heterogeneous
smoothness in different subregions of the entire domain. It is a more realistic
setting compared to homogeneous smoothness in the squared $\ell_2$ norm. The
divided differences with various orders realize the temporal evolution of
effective reproduction numbers with various degrees. 

We define a penalized regression to solve the MLE problem with the smoothness
regularization in \eqref{eq:rt-ptf}. It is a minimization problem with Poisson
loss (which is the negative log-likelihood of Poisson distributions) to control
the data fidelity and the trend filtering penalty to control the graphical
smoothness \citep{kim2009ell_1,tibshirani2014adaptive,tibshirani2022divided}.
The problem solves a Poisson trend filtering (PTF) estimator on univariate
cases. Let $\theta := \log(\calR) \in \bbR^n$, and then $\Lambda\circ \calR =
\Lambda\circ e^{\theta}$, $\log(\Lambda\circ \calR) = \log(\Lambda) + \theta$,
where $\circ$ is elementwise product, $e^{a}, \log(a)$ apply to vector $a$
elementwise. Let $w:=\Lambda$ to represent weights in the objective. Define the
problem with evenly spaced incidences as: 
%We further regularize the smoothness of the reproduction number using the $\ell_1$ norm of the divided difference of the natural logarithm of $\calR$, which is real-valued. 
\begin{equation} \label{eq:rt-ptf}
    \begin{split}
        \hat{\theta} &:= \Argmin{\theta\in\bbR^n} \fr{n}\sumN -y_i \theta_i + w_i e^{\theta_i} + \lambda \norm{D^{(k+1)} \theta}_1,         
    \end{split}
\end{equation}
where $D^{k+1} \in \bbZ^{(n-k-1)\times n}$ is a $(k+1)$st order divided difference matrix with $k = 0,1,2,\dots, n-2$, and $\hat{\calR} := e^{\hat{\theta}}$ solves the estimated effective reproduction numbers. Define $D^{(k+1)}$ recursively as $D^{(k+1)} := D^{(1)} D^{(k)}$, where $D^{(1)} \in \{-1,0,1\}^{(n-k-1)\times (n-k)}$ is a banded matrix of dynamic dimensions with diagonal band $(-1,1)$ and off-band components $0$s: 
$$D^{(1)} := 
\begin{pmatrix}
-1 & 1 &  & & \\
 & -1 & 1 & & \\
 & & \ddots & \ddots & \\
 & & & -1 & 1
\end{pmatrix}.
$$ 

Define $D^{(0)} := I_n$, which is an identity matrix with size $n$. %An exponential transformation is applied to the PTF estimator $\hat{\theta}$ to get the estimated reproduction numbers. 

The tuning parameter $\lambda$ balances the contributions between data fidelity and smoothness. When $\lambda=0$, the problem in \eqref{eq:rt-ptf} reduces to the regular least squares problem. A larger tuning parameter gives a higher importance on the regularization term and yields a smoother curve until the divided differences are all zeros, i.e., all parameters are projected onto the null space of the corresponding divided difference matrix(, since the tuning parameter is large enough to make the penalty term dominate the objective). 

For unevenly spaced observations, the distances between neighboring parameters vary by the time lengths between observation times, and thus, the divided differences should be adjusted by the days that the incidences are confirmed (i.e., data locations). Given the data locations $x_{1:n} = \{x_1,\dots,x_n\}$, for $k \geq 1$, define a $k$th order diagonal matrix $$X^k := \diag \lr{\frac{k}{x_{k+1} - x_1}, \frac{k}{x_{k+2} - x_2}, \cdots, \frac{k}{x_n - x_{n-k}} }.$$ Let $D^{(x,1)} := D^{(1)}$. For $k\geq 1$, define the $(k+1)$st order divided difference matrix for unevenly spaced incidences recursively as $$D^{(x,k+1)} := D^{(1)}\cdot X^k \cdot D^{(x,k)}.$$ 


Our estimator is locally adaptive so that it captures the local changes such as the initiation of effective control measures. More specifically, it regularizes the similarity among reproduction numbers across a chosen number of neighboring time points and segments the curvature of the reproduction numbers such that there are more jumpiness in some subregions and more smoothness in others. 
\cite{abry2020spatial,pascal2022nonsmooth} considered the second-order divided difference of effective reproduction numbers. In comparison to their studies, our estimator is more flexible in the degree of temporal evolution of the effective reproduction numbers and also avoids the potential numerical issues of penalizing/estimating positive real values. 

\subsection{Proximal Newton solver} %Specialized ADMM for `generalized' Poisson trend filtering on lines

The proximal Newton method is a second-order algorithm solving a proximal optimization iteratively followed by a line search algorithm adjusting the step size at each iteration for faster convergence. The proximal Newton method for Poisson trend filtering in \eqref{eq:rt-ptf} solves an approximate problem iteratively --- specifically, it takes a second-order Taylor expansion of the Poisson loss, which results in a proximal optimization, i.e., trend filtering with squared $\ell_2$ loss, with dynamic weights during iteration, and solves it iteratively until convergence to the objective. 

Let $g(\theta):= \fr{n} \sumN -y_i\theta_i + w_i e^{\theta_i}$ be the Poisson loss and $h(\theta) := \lambda \norm{D^{(k+1)} \theta}_1$ be the regularization in \eqref{eq:rt-ptf}. At iterate $t+1$, consider the following approximation of $g(\theta)$ using the second-order Taylor expansion around $\theta^t$, 
$$ g(\theta) = g(\theta^t) + (\theta - \theta^t)^{\top} \nabla^{(1)}_{\theta} g(\theta^t) + \fr{2} (\theta - \theta^t)^{\top} \nabla^{(2)}_{\theta} g(\theta^t) (\theta - \theta^t), $$
where $\nabla^{(1)}_{\theta} g(\theta^t) = \fr{n} \lr{-y + w\circ e^{\theta^t}} \in \bbR^n$ is the gradient of $g(\theta)$ at $\theta^t$ and $\nabla^{(2)}_{\theta} g(\theta^t) = \fr{n}\diag \lr{w \circ e^{\theta^t}} \in \bbR^{n\times n}$ is the Hessian matrix of $g(\theta)$ at $\theta^t$. %The gradient of $g(\theta)$ then can be approximated by $$\nabla_{\theta} g(\theta) := \nabla_{\theta} g(\theta^t) + \nabla^2_{\theta} g(\theta^t) (\theta - \theta^t) = \fr{n} \lr{W^t \theta - c^t} = \fr{n}W^t \lr{\theta - {c^t}^{\ast}},$$ where $c^t := y-e^{\theta^t}+\theta^t\circ e^{\theta^t}$, $W^t = \mathrm{diag}\lr{e^{\theta^t}}$, and ${c^t}^{\ast} = y\circ e^{-\theta^t} - \boldsymbol{1} + \theta^t$ given $e^{\theta_i^t} \in \bbR_{++}, i=1,...,n$. 

Define the proximal operator as $\mathrm{prox}_{W,D} (x) := \Argmin{z\in\bbR^n} \fr{2n} \norm{z-x}_W^2 + \lambda \norm{D\theta}_1$, where $\norm{a}_{W}^2 := a^{\top} {W} a$. The proximal optimization problem at iterate $t+1$ can be further written as, given $\theta^t$,
\begin{equation} \label{eq:prox-gauss}
    \begin{split}
        \theta^{t_+} :&= \Argmin{\theta\in\bbR^n} (\theta - \theta^t)^{\top} \nabla^{(1)}_{\theta} g(\theta^t) + \fr{2} (\theta - \theta^t)^{\top} \nabla^{(2)}_{\theta} g(\theta^t) (\theta - \theta^t) + h(\theta), \\
        &= \Argmin{\theta\in\bbR^n} \fr{2n} \norm{\theta - c^{t}}_{W^t}^2 + \lambda \norm{D^{(k+1)}\theta}_1, \\
        &= \mathrm{prox}_{W^t,D^{(k+1)}} (c^{t}),
    \end{split}
\end{equation}
where $W^t := \diag \lr{w\circ e^{\theta^t}}$ is the weighted (Hessian) matrix multiplied by $n$ and ${c^t} := \theta^t - n \lr{W^{t}}^{-1} \nabla^{(1)}_{\theta} g(\theta^t) = y\circ w^{-1}\circ e^{-\theta^t} - \boldsymbol{1} + \theta^t\circ w^{-1}$, where $\{e^{\theta^t}\}_{i\in[n]} > 0, [n]:=1,2,\dots,n$.
This is just univariate trend filtering with weights $W^t$ \citep{tibshirani2014adaptive}. 

We solve the trend filtering problem in \eqref{eq:prox-gauss} using the specialized ADMM, proposed by \cite{ramdas2016fast}, with the primal $\theta$ step solved in closed-form and the auxiliary step solved by the dynamic programming algorithm for fused lasso proposed by \cite{johnson2013dynamic}. Let the auxiliary variable $z:= D^{(k)}\theta$. The scaled augmented Lagrangian is $$\mathcal{L}_{\lambda, \rho}(\theta, z, u) = \fr{2n} \norm{\theta - c^{t}}_{W^t}^2 + \lambda \norm{D^{(1)}z}_1 + \frac{\rho}{2} \norm{D^{(k)}\theta - z + u}^2 - \frac{\rho}{2} \norm{u}^2, $$ where $\rho$ is a scaled dual parameter and $u$ is a dual variable. At Newton's iteration $t+1$, the specialized ADMM solves the following subproblems, at ADMM iteration $l+1$: 
\begin{equation}
  \begin{split}
      \theta^{l+1} &:= \Argmin{\theta} \fr{2n} \norm{\theta - c^{t}}_{W^t}^2 + \frac{\rho}{2} \norm{D^{(k+1)} \theta - z^l + u^l}_2^2, \\
      z^{l+1} &:= \Argmin{z} \frac{\lambda}{\rho} \norm{D^{(1)} z}_1 + \fr{2} \norm{D^{(k+1)} \theta^{l+1} - z + u^l}_2^2, \\
      u^{l+1} &\leftarrow u^l + D^{(k+1)} \theta^{l+1} - z^{l+1}.
  \end{split}
\end{equation}

We further adjust the step size $\gamma^{t+1} \in (0,1]$ at iterate $t+1$ by a backtracking line search algorithm to solve for $\theta^{t+1}$, i.e.,   
$$\theta^{t+1} \leftarrow \theta^t + \gamma^{t+1} (\theta^{t_+} - \theta^t).$$ The proximal Newton algorithm iterates until convergence of the objective.

% defer to the section of Simulation: 
%Range of $\lambda$: Find the maximum lambda such that the estimated $\theta$ does not fall into the null space of the divided difference matrix.
%% other computational considerations: low quality data (missingness, outliers, seasonalities, ...)? 

\subsection{Bayesian perspective}
%The epidemiology evolution mechanism suggests a hierarchical framework that posteriori of reproduction number depends on its prior and the distribution of confirmed cases. 

%We here provide an alternative interpretation of our approach from the Bayesian perspective. 
Our approach can be interpreted as a state-space model of Poisson observational noises and Laplace transition noises with certain degree $k\geq 0$, e.g., $\theta_{t+1} = 2\theta_t - \theta_{t-1} + \varepsilon_{t+1}$ with $\varepsilon_{t+1}\sim \mathrm{Laplace}(0,1/\lambda)$ for $k=1$. Compared to EpiFilter \citep{parag2021improved}, another retrospective study of $\calR_t$, we share same observational assumptions, but our approach has a different transition noises. 
EpiFilter estimates the posterior distribution of $\calR_t$, and thus it can provide the credible interval estimation with various credible levels. Our approach solves the point estimation using optimization problem, which has the advantage of computational efficiency. 

\section{Results}

Implementation of the proximal Newton method is provided in the \texttt{R} package \href{https://dajmcdon.github.io/rtestim/}{\texttt{rtestim}}. 

\subsection{Covid-19 cases}

% introduce data & hyperparameter setup
We implement the proposed model on the Covid-19 confirmed cases in British Columbia (B.C.) as of May 18, 2023 reported by B.C. Conservation Data Centre. We choose the gamma distribution with shape $2.5$ and scale $2.5$ to approximate the serial interval function.

\begin{figure}[tb]
    \centering
    %\includegraphics[width=0.99\linewidth]{fig/covid19fig.pdf}
    \caption{Covid19 daily confirmed counts between March 1st, 2020 and April 15th, 2023 in British Columbia, Canada. The top left panel displays the time trend of the observed infectious cases. The top right, bottom left and bottom right panels illustrated the estimated reproduction numbers ($\calR_t$) using the Poisson trend filtering (in \eqref{eq:rt-ptf}) with degrees $k=1,2,3$ respectively.} 
\end{figure}

% interpret figures -- across all lambdas
Considering the temporal evolutions of neighboring $3, 4, 5$ reproduction numbers, the estimated reproduction numbers of Covid-19 in British Columbia (displayed in the top right, bottom left, and bottom right panels in Fig 1 respectively) are always lower than $2.5$, which means that two distinct infected individuals can on average infect less than five other individuals in the population. The three degrees of the temporal evolution (across all regularization levels $\lambda$) all yield similar results that $\hat{\calR}_t$ achieves the highest peak around the end of 2021 and reaches the lowest trough shortly thereafter. Throughout the estimated curves, the peaks and troughs of the reproduction numbers roughly come prior to the following growths and decays of confirmed cases respectively.

The reproduction numbers are relatively unstable before April 1st, 2022.
The highest peak coincides with the emergence and globally spread of the Omicron variant. The estimated reproduction numbers are apparently below the threshold $1$ during two time periods -- roughly from April 1st, 2021 to July 1st, 2021 and from January 1st, 2022 to April 1st, 2022. The first trough of $\hat{\calR}_t$ coincides with the first authorization for use of Covid-19 vaccines in British Columbia. The second trough shortly after the greatest peak may credit to many aspects, including self-isolation of the infected individuals and application of the second shot of Covid-19 vaccines. 
Since around April 1st, 2022, the reproduction numbers stay stable (at around $1$) and the infected cases stay low. 

% for different lambda
Greater regularization levels (i.e., larger $\lambda$s) result in smoother estimated curves. Smoother curves (e.g., the yellow curve in the top right panel in Fig 1) suggest that the estimated reproduction numbers are around $1$ during most time periods; however, they may not be appropriate to interpret the reality. More wiggly curves better reflect the fluctuation of $\calR_t$, but sometimes fail to highlight the significant peaks or troughs. The tuning parameter $\lambda$ needs to be chosen corresponding to the information in practice for a better interpretation.

\section{Discussion}

% advantages
\RtEstim\ provides a locally adaptive estimator using Poisson trend filtering on univariate data. It captures the heterogeneous smoothness of effective reproduction numbers given the observed incidence time series in a certain region. This is a nonparametric regression model which can be written as a convex optimization(minimization) problem. Minimizing the distance (averaged KL divergence per coordinate) between the estimators and (functions of) observations guarantees the data fidelity; minimizing a certain order of divided differences between each pair of neighboring parameters regularizes the smoothness. The $\ell_1$ regularization introduces sparsity to the divided differences, which leads to heterogeneous smoothness within certain periods of time. The homogeneous smoothness within a time period can be either performed by a constant reproduction number, or a constant rate of changes, or a constant graphical curvature depending on the prescribed degree ($k=0,1,2$ respectively). %The estimator is uniquely defined due to its strict convexity. 

The property of local adaptivity is useful to distinguish, for example, the seasonal outbreaks from the un-seasonal outbreak periods. Given a properly chosen degree of polynomials, for example, the growth rate of un-seasonal outbreak periods can suggest a potential upcoming outbreak, which alerts epidemiologists to propose sanitary policies to prevent the progressing outbreak ahead of the infection surge. The effective reproduction numbers can be estimated afterwards to check the efficiency of the sanitary policies referring to whether they are below the threshold, their tendencies of reduction, or their graphical curvatures.

% assumptions and limitations
Our method \RtEstim\ provides a natural way to deal with missing data, e.g., on weekends and holidays. We linearly impute the missing data in the computation of total primary infectiousness by assuming these values are missing at random. 
While solving the convex optimization problem, the edge lengths of the line graphs can be adjusted, so we can manually increase the length between two observations while penalizing the distance between them. 
% Moreover, the $\ell_1$ penalty introduces sparsity into curvature, and thus, makes the estimators less sensitive to outliers compared to $\ell_2$ regularization. 
It is remarkable that our focus is to provide a mathematical model for epidemiologists to use, rather than to focus on a specific disease. In addition, more specialized methodologies are needed for the diseases with relatively long incubation periods (e.g., HIV and HBV). 
% used for acute diseases with short serial intervals 


% most generally about Rt
% can vary significantly due to different assumptions; how to choose among all methods? how to choose the hyperparameters in an individual model? 
% should it be more smooth or more wiggly? 
% may depend on the purposes: interpret information (smoother?) or forecasting (more wiggly/joggled?) 
%Hence, mathematical modelling requires the domain expertise for specific infectious diseases dynamics to set the assumptions. % varying along with local sociobehavioral and environmental circumstances.

% about retrospective v.s. real-time?



% an existing approach to model R_0
A group of epidemiological models are compartmental models. They establish the epidemic transmission process by creating compartments with labels and connecting them by directed edges. A simple compartmental model -- for example, \textit{Susceptible-Infectious-Susceptible} (SIS) model -- divides the population ($N$) into two compartments for susceptible cases ($S$) and infectious cases ($I$) respectively and connects them in serial as $S\to I\to S$. It only focuses on susceptible individuals. Each directed edge corresponds to a ratio of transmission (say, $\alpha,\beta$ respectively). In such models, reproduction numbers are defined as functions of the estimated transmission parameters and the numbers of compartments or population, e.g., $\hat{\calR}_0=\hat{\beta} N/\hat{\alpha}$ in the SIS models \cite{brauer2019mathematical}, as by-products. Compartmental models usually solve ordinary differential equations (ODE) systems for transmission numbers (e.g., $\alpha,\beta$ in the SIS model). A disadvantage of such parametric models is that they are less flexible than nonparametric models and the number of parameters to be estimated grows along with the increase of compartments in practice, which results in a growing computational complexity. Since the epidemic mechanism depends highly on the contexts, e.g., if a latency period exists or not, such models are lack of generalizability. Moreover, data of high quality are not always available for all compartments especially when there is a pandemic outbreak that results in a sudden shortage of resources in collecting daily new infections. %the sensitivity to low-quality data and the complex computation
%To overcome these limitations of compartmental models, there is an alternative branch of approaches estimating reproduction numbers directly without expressing them as functions of other transmission parameters. 

% other obstacles of Rt estimation in \cite{gostic2020practical}
%such as the difference between generation time and serial interval (given constant incubation time), right consored estimates? ... 
% cite other papers exploring approaches to solve these problems.



% limitation
There are more practical considerations that may influence the quality of $\calR_t$ estimation to be considered late. 
In our approach, we consider a homogeneous population without distinguishing the imported cases from the local cases. 
% NB alternatives 
Poisson distribution is frequently used to model non-negative count data with heteroskedasticity. Another common alternative is negative Binomial distribution with or without a specified level of overdispersion. 
% 
We consider a fixed serial interval throughout the transmission dynamics, but as the factors such as population immunity vary, the serial interval may vary as well due to the change of population factors such as herd immunity. % (\citep{nash2023estimating} pp.12). 
Another common statement is that the distribution density of serial intervals is generally wider than the correspondence of generation intervals as serial interval includes both generation time and incubation time. If we assume generation time and incubation time both follow gamma distributions, the serial interval is likely to perform as a bimodal density. 

% how about regional evolution?
%A limitation is that it only takes the temporal evolution within a single region into account, and fails to consider the spatial connection or spatial-temporal evolution among regions. 
 %A potential future work is to extend the proposed model to analyze spatio-temporal transmission data. Such data has the inherit graphical structure such that temporal evolution within a region can be connected by lines (as time series) and spatial connection (of cross-sectional data) can be constructed by graphs where each pair of neighboring regions is linked by an edge. Moreover, the spatio-temporal evolution, i.e., the effects of previous infectious data of one region on current infections of neighboring regions, can be measured, for example, by linking the node of region $a$ at time $t-1$ to another node of region $b$ at time $t$. 
%Through this way, different orders of spatial and temporal evolutions can be manually manipulated. 
 %In this case, we can directly apply Poisson trend filtering on graphs with minor adjustment. %A remarkable note is that the edge lengths need to be made comparable across temporal and regional connections.


%\section*{Acknowledgement}


\addcontentsline{toc}{section}{References}

\bibliographystyle{doc/rss.bst}
\bibliography{doc/ptf}

%\appendix
%\titleformat{\chapter}{\normalfont\huge\bfseries}{Appendix \thechapter}{1em}{}

\end{document}
